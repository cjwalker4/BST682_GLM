\documentclass[]{article}
\usepackage{lmodern}
\usepackage{amssymb,amsmath}
\usepackage{ifxetex,ifluatex}
\usepackage{fixltx2e} % provides \textsubscript
\ifnum 0\ifxetex 1\fi\ifluatex 1\fi=0 % if pdftex
  \usepackage[T1]{fontenc}
  \usepackage[utf8]{inputenc}
\else % if luatex or xelatex
  \ifxetex
    \usepackage{mathspec}
  \else
    \usepackage{fontspec}
  \fi
  \defaultfontfeatures{Ligatures=TeX,Scale=MatchLowercase}
\fi
% use upquote if available, for straight quotes in verbatim environments
\IfFileExists{upquote.sty}{\usepackage{upquote}}{}
% use microtype if available
\IfFileExists{microtype.sty}{%
\usepackage{microtype}
\UseMicrotypeSet[protrusion]{basicmath} % disable protrusion for tt fonts
}{}
\usepackage[margin=1in]{geometry}
\usepackage{hyperref}
\hypersetup{unicode=true,
            pdftitle={Homework 1 - BST682},
            pdfborder={0 0 0},
            breaklinks=true}
\urlstyle{same}  % don't use monospace font for urls
\usepackage{graphicx,grffile}
\makeatletter
\def\maxwidth{\ifdim\Gin@nat@width>\linewidth\linewidth\else\Gin@nat@width\fi}
\def\maxheight{\ifdim\Gin@nat@height>\textheight\textheight\else\Gin@nat@height\fi}
\makeatother
% Scale images if necessary, so that they will not overflow the page
% margins by default, and it is still possible to overwrite the defaults
% using explicit options in \includegraphics[width, height, ...]{}
\setkeys{Gin}{width=\maxwidth,height=\maxheight,keepaspectratio}
\IfFileExists{parskip.sty}{%
\usepackage{parskip}
}{% else
\setlength{\parindent}{0pt}
\setlength{\parskip}{6pt plus 2pt minus 1pt}
}
\setlength{\emergencystretch}{3em}  % prevent overfull lines
\providecommand{\tightlist}{%
  \setlength{\itemsep}{0pt}\setlength{\parskip}{0pt}}
\setcounter{secnumdepth}{0}
% Redefines (sub)paragraphs to behave more like sections
\ifx\paragraph\undefined\else
\let\oldparagraph\paragraph
\renewcommand{\paragraph}[1]{\oldparagraph{#1}\mbox{}}
\fi
\ifx\subparagraph\undefined\else
\let\oldsubparagraph\subparagraph
\renewcommand{\subparagraph}[1]{\oldsubparagraph{#1}\mbox{}}
\fi

%%% Use protect on footnotes to avoid problems with footnotes in titles
\let\rmarkdownfootnote\footnote%
\def\footnote{\protect\rmarkdownfootnote}

%%% Change title format to be more compact
\usepackage{titling}

% Create subtitle command for use in maketitle
\newcommand{\subtitle}[1]{
  \posttitle{
    \begin{center}\large#1\end{center}
    }
}

\setlength{\droptitle}{-2em}

  \title{Homework 1 - BST682}
    \pretitle{\vspace{\droptitle}\centering\huge}
  \posttitle{\par}
    \author{}
    \preauthor{}\postauthor{}
      \predate{\centering\large\emph}
  \postdate{\par}
    \date{Assigned: August 31, 2018}


\begin{document}
\maketitle

{
\setcounter{tocdepth}{3}
\tableofcontents
}
\subsubsection{Homework 1 Overview}\label{homework-1-overview}

This homework is intended to serve three main purposes: (1) familiarize
you with our course homework submission policy (while satisfying the new
Title IV regulations), (2) refresh your probability and linear modeling
skills, and (3) introduce you to R. Give complete solutions, justifying
your response when necessary.

This homework is due by \textbf{9:30am on Tuesday, September 11}. To
complete this assignment, follow these steps:

\begin{enumerate}
\def\labelenumi{\arabic{enumi}.}
\item
  Answer the questions below in a format for which you are comfortable
  (e.g., LaTeX, \(\texttt{R}\), Word, paper, etc.).
\item
  Convert this work to a pdf and name it lastname1.pdf (replacing
  `lastname' with your last name in lowercase).
\item
  For any questions that require programming, provide a similarly-named
  file that includes fully-reproducible code. (e.g., lastname1.r,
  lastname1.rmd or lastname1.sas)
\item
  Submit these files to Canvas.
\end{enumerate}

\subsubsection{Problem 1: probablity refresher
1}\label{problem-1-probablity-refresher-1}

Uber, AirBnb and Stata have 3000, 1500, and 800 employees, respectively,
and 30, 45, and 65 percent of these employees respectively are women.
Resignations are equally likely among companies and genders. One woman
resigns. What is the probability she worked for Uber?

\subsubsection{Problem 2: probability refresher
2}\label{problem-2-probability-refresher-2}

You flip four fair coins. Assuming the flips are independent, what is
the pmf for the number of tails flipped?

\subsubsection{Problem 3: probability refresher
3}\label{problem-3-probability-refresher-3}

Do problem 1.6 (a,b) from our text.

\subsubsection{Problem 4: probability refresher
4}\label{problem-4-probability-refresher-4}

Assume annual rainfall in Lexington is normally distributed with a mean
of 40 inches and standard deviation of 4. What is the probability that
it takes more than 7 years before having a rainfall over 55 inches? What
assumptions are you making?

\subsubsection{Problem 5: linear models
refresher}\label{problem-5-linear-models-refresher}

Using the data from
\texttt{Table\ 2.3\ Birthweight\ and\ gestational\ age.xls}, calculate
by matrix algebra the effect estimate resulting from regressing birth
weight on gestational age.

\subsubsection{Problem 6: R intro 1}\label{problem-6-r-intro-1}

You will inevitably use the Google to problem solve with programming in
R -- many of you already do. Having go to resources for answering your
questions and/or developing new skills can be quite helpful. Search
around for what might be (or already is) a resource you will turn to as
you improve your R skills. Give the site and url. What, in particular,
makes this suitable for you?

\subsubsection{Problem 7: R intro 2}\label{problem-7-r-intro-2}

Import the data from
\texttt{Table\ 2.3\ Birthweight\ and\ gestational\ age.xls} into R. Each
observation should be a single row. \emph{Tip}: I added a second sheet
to make this easier if you prefer. Use the \texttt{Import\ Dataset}
functionality in RStudio's \texttt{Environment} tab and select Sheet 2.
This simple example shows why some abhor Excel\ldots{} \emph{Tip 2}: Use
the \texttt{readxl} package.

Plot birthweight by age and give each gender a different color on the
same plot. Now, do the same plot stratified by gender (Tip: look at the
Introduction to R notes). What observations do you have?

\subsubsection{Problem 8: R intro 3}\label{problem-8-r-intro-3}

Using \texttt{R} and \texttt{lm}, confirm your regression parameter
estimate in Problem 5.


\end{document}
